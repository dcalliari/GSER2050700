%% abtex2-modelo-trabalho-academico.tex, v-1.9.2 laurocesar
%% Copyright 2012-2014 by abnTeX2 group at http://abntex2.googlecode.com/ 
%%
%% This work may be distributed and/or modified under the
%% conditions of the LaTeX Project Public License, either version 1.3
%% of this license or (at your option) any later version.
%% The latest version of this license is in
%%   http://www.latex-project.org/lppl.txt
%% and version 1.3 or later is part of all distributions of LaTeX
%% version 2005/12/01 or later.
%%
%% This work has the LPPL maintenance status `maintained'.
%% 
%% The Current Maintainer of this work is the abnTeX2 team, led
%% by Lauro César Araujo. Further information are available on 
%% http://abntex2.googlecode.com/
%%
%% This work consists of the files abntex2-modelo-trabalho-academico.tex,
%% abntex2-modelo-include-comandos and abntex2-modelo-references.bib
%%

% ------------------------------------------------------------------------
% ------------------------------------------------------------------------
% abnTeX2: Modelo de Trabalho Academico (tese de doutorado, dissertacao de
% mestrado e trabalhos monograficos em geral) em conformidade com 
% ABNT NBR 14724:2011: Informacao e documentacao - Trabalhos academicos -
% Apresentacao
% ------------------------------------------------------------------------
% ------------------------------------------------------------------------

%-------------------------------------------------------------------------
% Modelo adaptado especificamente para o contexto do PPgSI-EACH-USP por 
% Marcelo Fantinato, com auxílio dos Professores Norton T. Roman, Helton
% H. Bíscaro e Sarajane M. Peres, em 2015, com muitos agradecimentos aos 
% criadores da classe e do modelo base.
%
% 20/06/2017: inclusão de "lista de quadros" com base no especificado em:
% https://github.com/abntex/abntex2/wiki/HowToCriarNovoAmbienteListing,
% de autoria de "Eduardo de Santana Medeiros Alexandre".
%
%-------------------------------------------------------------------------

\documentclass[
	% -- opções da classe memoir --
	12pt,				% tamanho da fonte
	% openright,			% capítulos começam em pág ímpar (insere página vazia caso preciso)
	oneside,			% para impressão apenas no anverso (apenas frente). Oposto a twoside
	a4paper,			% tamanho do papel. 
	% -- opções da classe abntex2 --
	%chapter=TITLE,		% títulos de capítulos convertidos em letras maiúsculas
	%section=TITLE,		% títulos de seções convertidos em letras maiúsculas
	%subsection=TITLE,	% títulos de subseções convertidos em letras maiúsculas
	%subsubsection=TITLE,% títulos de subsubseções convertidos em letras maiúsculas
	% -- opções do pacote babel --
	english,			% idioma adicional para hifenização
	%french,				% idioma adicional para hifenização
	%spanish,			% idioma adicional para hifenização
	brazil				% o último idioma é o principal do documento
	]{abntex2unama}

% ---
% Pacotes básicos 
% ---
\usepackage{lmodern}			% Usa a fonte Latin Modern			
\usepackage[T1]{fontenc}		% Selecao de codigos de fonte.
\usepackage[utf8]{inputenc}		% Codificacao do documento (conversão automática dos acentos)
\usepackage{lastpage}			% Usado pela Ficha catalográfica
\usepackage{indentfirst}		% Indenta o primeiro parágrafo de cada seção.
\usepackage{color}				% Controle das cores
\usepackage{graphicx}			% Inclusão de gráficos
\usepackage{microtype} 			% para melhorias de justificação
\usepackage{pdfpages}     %para incluir pdf
\usepackage{algorithm}			%para ilustrações do tipo algoritmo
\usepackage{mdwlist}			%para itens com espaço padrão da abnt
\usepackage[noend]{algpseudocode}			%para ilustrações do tipo algoritmo
		
% ---
% Pacotes adicionais, usados apenas no âmbito do Modelo Canônico do abnteX2
% ---
\usepackage{lipsum}				% para geração de dummy text
% ---

% ---
% Pacotes de citações
% ---
\usepackage{hyperref}
\usepackage[brazilian,hyperpageref]{backref}	 % Paginas com as citações na bibl
\usepackage[alf,abnt-etal-list=0,abnt-etal-text=it]{abntex2cite}	% Citações padrão ABNT

% --- 
% CONFIGURAÇÕES DE PACOTES
% --- 

% ---
% Configurações do pacote backref
% Usado sem a opção hyperpageref de backref
\renewcommand{\backrefpagesname}{Citado na(s) página(s):~}
% Texto padrão antes do número das páginas
\renewcommand{\backref}{}
% Define os textos da citação
\renewcommand*{\backrefalt}[4]{
	\ifcase #1 %
		Nenhuma citação no texto.%
	\or
		Citado na página #2.%
	\else
		Citado #1 vezes nas páginas #2.%
	\fi}%
% ---

% ---
% Informações de dados para CAPA e FOLHA DE ROSTO
% ---

%-------------------------------------------------------------------------
% Comentário adicional do PPgSI - Informações sobre o ``instituicao'':
%
% Não mexer. Deixar exatamente como está.
%
%-------------------------------------------------------------------------
\instituicao{
	UNIVERSIDADE DA AMAZÔNIA
	\par
	CURSO DE BACHARELADO EM CIÊNCIA DA COMPUTAÇÃO
	\par
	DISCIPLINA DE INFRAESTRUTURA DE DATACENTERS
	}

%-------------------------------------------------------------------------
% Comentário adicional do PPgSI - Informações sobre o ``título'':
%
% Em maiúscula apenas a primeira letra da sentença (do título), exceto 
% nomes próprios, geográficos, institucionais ou Programas ou Projetos ou 
% siglas, os quais podem ter letras em maiúscula também.
%
% O subtítulo do trabalho é opcional.
% Sem ponto final.
%
% Atenção: o título da Dissertação/Tese na versão corrigida não pode mudar. 
% Ele deve ser idêntico ao da versão original.
%
%-------------------------------------------------------------------------
\titulo{Projeto Conceitual de um Data Center de Alta Disponibilidade Tier III}

%-------------------------------------------------------------------------
% Comentário adicional do PPgSI - Informações sobre o ``autor'':
%
% Todas as letras em maiúsculas.
% Nome completo.
% Sem ponto final.
% Para adicionar mais de um autor, basta acrescentar \\ entre os autores.
%-------------------------------------------------------------------------
\autor{\uppercase{Daniel Bahia Pinheiro Calliari}}

%-------------------------------------------------------------------------
% Comentário adicional do PPgSI - Informações sobre o ``local'':
%
% Não incluir o ``estado''.
% Sem ponto final.
%-------------------------------------------------------------------------
\local{Belém}

%-------------------------------------------------------------------------
% Comentário adicional do PPgSI - Informações sobre a ``data'':
%
% Colocar o ano do depósito (ou seja, o ano da entrega) da respectiva 
% versão, seja ela a versão original (para a defesa) seja ela a versão 
% corrigida (depois da aprovação na defesa). 
%
% Atenção: Se a versão original for depositada no final do ano e a versão 
% corrigida for entregue no ano seguinte, o ano precisa ser atualizado no 
% caso da versão corrigida. 
% Cuidado, pois o ano da ``capa externa'' também precisa ser atualizado 
% nesse caso.
%
% Não incluir o dia, nem o mês.
% Sem ponto final.
%-------------------------------------------------------------------------
\data{2025}

%-------------------------------------------------------------------------
% Comentário adicional do PPgSI - Informações sobre o ``Orientador'':
%
% Se for uma professora, trocar por ``Profa. Dra.''
% Nome completo.
% Sem ponto final.
%-------------------------------------------------------------------------
\orientador{Prof. Dr. Fulano de Tal}

%-------------------------------------------------------------------------
% Comentário adicional do PPgSI - Informações sobre o ``Coorientador'':
%
% Opcional. Incluir apenas se houver co-orientador formal, de acordo com o 
% Regulamento do Programa.
%
% Se for uma professora, trocar por ``Profa. Dra.''
% Nome completo.
% Sem ponto final.
%-------------------------------------------------------------------------
\coorientador{Prof. Dr. Fulano de Tal}

\tipotrabalho{Dissertação (Mestrado) / Tese (Doutorado)}

\preambulo{
%-------------------------------------------------------------------------
% Comentário adicional do PPgSI - Informações sobre o texto ``Versão 
% original'':
%
% Não usar para Qualificação.
% Não usar para versão corrigida de Dissertação/Tese.
%
%-------------------------------------------------------------------------
Versão original \newline \newline \newline 
%-------------------------------------------------------------------------
% Comentário adicional do PPgSI - Informações sobre o ``texto principal do
% preambulo'':
%
% Para Doutorado, trocar por: Tese apresentada à Escola de Artes, Ciências e Humanidades da Universidade de São Paulo para obtenção do título de Doutor (ou Doutora) em Ciências pelo Programa de Pós-graduação em Sistemas de Informação. 
%
% Para Qualificação, trocar por: Projeto de pesquisa para exame de qualificação apresentado à Escola de Artes, Ciências e Humanidades da Universidade de São Paulo como parte dos requisitos para obtenção do título de Mestre (ou Doutor ou Doutora) em Ciências pelo Programa de Pós-graduação em Sistemas de Informação.
%
%-------------------------------------------------------------------------
Trabalho acadêmico apresentado à Universidade da Amazônia como parte de disciplinas do Curso de Bacharelado em Ciência da Computação.
%
\newline \newline Área de concentração: Infraestrutura de Datacenters
%-------------------------------------------------------------------------
% Comentário adicional do PPgSI - Informações sobre o texto da ``Versão 
% corrigida'':
%
% Não usar para Qualificação.
% Não usar para versão original de Dissertação/Tese.
% 
% Substituir ``xx de xxxxxxxxxxxxxxx de xxxx'' pela ``data da defesa''.
%
%-------------------------------------------------------------------------
% \newline \newline \newline Versão corrigida contendo as alterações solicitadas pela comissão julgadora em xx de xxxxxxxxxxxxxxx de xxxx. A versão original encontra-se em acervo reservado na Biblioteca da EACH-USP e na Biblioteca Digital de Teses e Dissertações da USP (BDTD), de acordo com a Resolução CoPGr 6018, de 13 de outubro de 2011.
}
% ---


% ---
% Configurações de aparência do PDF final

% alterando o aspecto da cor azul
\definecolor{blue}{RGB}{41,5,195}

% informações do PDF
\makeatletter
\hypersetup{
     	%pagebackref=true,
		pdftitle={\@title}, 
		pdfauthor={\@author},
    	pdfsubject={\imprimirpreambulo},
	    pdfcreator={laTeX com abnTeX2 adaptado para a UNAMA/ALC},
		pdfkeywords={abnt}{latex}{abntex}{abntex2unama}{trabalho acadêmico}{unama}, 
		colorlinks=true,       		% false: boxed links; true: colored links
    	linkcolor=blue,          	% color of internal links
    	citecolor=blue,        		% color of links to bibliography
    	filecolor=magenta,      		% color of file links
		urlcolor=blue,
		bookmarksdepth=4
}
\makeatother
% --- 

% --- 
% Espaçamentos entre linhas e parágrafos 
% --- 

% O tamanho do parágrafo é dado por:
\setlength{\parindent}{1.25cm}

% Controle do espaçamento entre um parágrafo e outro:
\setlength{\parskip}{0cm}  % tente também \onelineskip
\renewcommand{\baselinestretch}{1.5}

% ---
% compila o indice
% ---
\makeindex
% ---

	% Controlar linhas orfas e viuvas
  \clubpenalty10000
  \widowpenalty10000
  \displaywidowpenalty10000

% ----
% Início do documento
% ----
\begin{document}

% Retira espaço extra obsoleto entre as frases.
\frenchspacing

% ----------------------------------------------------------
% ELEMENTOS PRÉ-TEXTUAIS
% ----------------------------------------------------------
% \pretextual

% ---
% Capa
% ---
%-------------------------------------------------------------------------
% Comentário adicional do PPgSI - Informações sobre a ``capa'':
%
% Esta é a ``capa'' principal/oficial do trabalho, a ser impressa apenas 
% para os casos de encadernação simples (ou seja, em ``espiral'' com 
% plástico na frente).
% 
% Não imprimir esta ``capa'' quando houver ``capa dura'' ou ``capa brochura'' 
% em que estas mesmas informações já estão presentes nela.
%
%-------------------------------------------------------------------------
\imprimircapa
% ---

% ---
% inserir o sumario
% ---
\pdfbookmark[0]{\contentsname}{toc}
\tableofcontents*
\cleardoublepage
% ---



% ----------------------------------------------------------
% ELEMENTOS TEXTUAIS
% ----------------------------------------------------------
\textual



%-------------------------------------------------------------------------
% Comentário adicional do PPgSI - Informações sobre ``títulos de seções''
% 
% Para todos os títulos (seções, subseções, tabelas, ilustrações, etc.):
%
% Em maiúscula apenas a primeira letra da sentença (do título), exceto 
% nomes próprios, geográficos, institucionais ou Programas ou Projetos ou
% siglas, os quais podem ter letras em maiúscula também.
%
%-------------------------------------------------------------------------
\chapter{Introdução}
Este trabalho apresenta o planejamento de um data center de alta disponibilidade com classificação Tier III. O objetivo principal é garantir a continuidade operacional, atendendo a SLAs rigorosos e proporcionando escalabilidade e segurança.

O projeto foi desenvolvido com base em padrões internacionais e melhores práticas, abordando aspectos como redundância, eficiência energética, segurança física e lógica, além de um layout otimizado para suportar as demandas atuais e futuras. A escolha do Tier III reflete um equilíbrio entre custo e benefício, garantindo alta disponibilidade (99.982\%) e capacidade de manutenção simultânea.

\chapter{Escolha do Tier e Justificativa}
\section{Tier III}
O Tier III foi escolhido por oferecer alta disponibilidade (99.982\%) e capacidade de manutenção simultânea, com um equilíbrio entre custo e benefício. Ele exige redundância N+1 em sistemas críticos e permite manutenção sem interrupção das operações. Embora o Tier IV ofereça maior robustez, o Tier III atende às necessidades do cenário proposto, garantindo um custo-benefício adequado.

O Tier IV é ainda mais robusto (considerado “fault-tolerant”), mas também muito mais caro e complexo de implantar. Em muitos cenários, o Tier III é suficiente para atender SLAs (Service Level Agreements) exigentes, mantendo um custo-benefício equilibrado.

\chapter{Estratégias de Redundância}
\section{Energia}
\subsection{Entrada de Energia}
Múltiplas entradas de energia da concessionária foram planejadas para garantir redundância. Caso não seja viável ter duas concessionárias distintas, serão utilizadas subestações ou alimentadores independentes.

\subsection{UPS (Uninterruptible Power Supply)}
Configuração N+1 será adotada, com cada UPS suportando a carga total e um adicional para redundância. As baterias serão dimensionadas para autonomia de 15 a 30 minutos, garantindo tempo suficiente para o startup dos geradores.

\subsection{Geradores}
Os geradores serão configurados em N+1, com capacidade para atender 100\% da carga crítica. Tanques de combustível terão autonomia mínima de 12 horas, idealmente 24 horas.

\section{Redes}
\subsection{ISPs (Internet Service Providers)}
Serão contratados dois ou mais provedores distintos, com roteamento BGP (quando possível), garantindo redundância de rota. Caminhos físicos de fibra serão diversificados, com entradas diferentes no prédio, sempre que possível.

\subsection{Equipamentos de Rede Internos}
Switches, roteadores e firewalls serão configurados em pares redundantes, utilizando tecnologias como HSRP/VRRP para roteadores e stack ou VSS para switches. O cabeamento estruturado será feito com CAT6A ou superior e fibra ótica, garantindo caminhos redundantes (Patch Panel A e B).

\subsection{Segmentação e Virtualização de Rede}
VLANs serão criadas para segmentar o tráfego (produção, gerência, storage, etc.). Sistemas de virtualização de rede (SDN) poderão ser utilizados para melhorar o gerenciamento e a flexibilidade, com atenção especial à redundância.

\section{Resfriamento}
\subsection{Sistemas de Climatização}
Projeto de N+1 nos equipamentos de ar-condicionado de precisão (CRAC/CRAH). Se utilizar água gelada, ter chillers redundantes com circuitos independentes.

\subsection{Distribuição de Ar}
Layout de corredores quentes e frios (“hot aisle/cold aisle”) para otimizar fluxo de ar e eficiência energética. Monitoramento de temperatura e umidade em vários pontos para identificar pontos de calor (hotspots).

\subsection{Controles e Alarmes}
Sistemas de controle ambiental integrados (BMS – Building Management System) para gerenciar temperatura, umidade e detecção de vazamentos. Alarmes e monitoramento 24/7, com notificação imediata em caso de falha em algum componente de refrigeração.

\chapter{Layout Conceitual}
\section{Disposição Física}
\subsection{Sala Principal de Servidores}
A sala principal será organizada com corredores quentes e frios. O corredor frio será onde ficam as portas frontais dos racks para admissão de ar frio, enquanto o corredor quente será onde o ar quente é expelido pela parte traseira dos racks. O espaço será planejado para comportar 10 racks inicialmente, com possibilidade de expansão futura.

\subsection{Racks}
Cada rack terá altura padrão de 42U e será dividido de forma lógica:
\begin{itemize}
	\item \textbf{Racks de rede:} switches de core/distribuição, roteadores e firewalls, normalmente posicionados nos primeiros racks para facilitar o cabeamento.
	\item \textbf{Racks de servidores:} servidores em blade ou rack mount.
	\item \textbf{Racks de storage:} equipamentos de armazenamento (SAN/NAS).
\end{itemize}
Será dada atenção especial à organização de cabos (cable management) e à separação de caminhos independentes para cabos de rede e de energia.

\subsection{Sala de Backup / Tape Library / Cofre}
Será ideal ter uma sala separada ou ao menos um espaço isolado para bibliotecas de fitas (tape library) e/ou equipamentos de backup em disco. Backups críticos serão mantidos em outro local (off-site) para contingência de desastres.

\subsection{Sala de UPS e Painéis Elétricos}
A sala de UPS será posicionada próxima à entrada de energia e distante da sala de servidores para evitar ruído e calor. O acesso será restrito e devidamente sinalizado.

\subsection{Sala de Geradores}
Os geradores serão instalados em uma sala externa ou isolada, com boa ventilação e segurança. Os tanques de combustível serão protegidos contra vazamentos e terão acesso controlado.

\section{Equipamentos Principais}
\subsection{Servidores}
Será possível optar por servidores do tipo blade ou rack. Para demandas de virtualização, serão priorizados servidores de alta densidade, com recursos robustos de CPU, memória e interfaces de rede.

\subsection{Storage}
O armazenamento será baseado em SAN (Storage Area Network) utilizando protocolos Fibre Channel ou iSCSI. Os equipamentos terão redundância, incluindo controladoras duplas, discos configurados em RAID e fontes redundantes.

\subsection{Sistemas de Backup}
O sistema de backup contará com software especializado (como Veeam, CommVault ou Bacula) rodando em servidores dedicados. Poderão ser utilizadas tape libraries para retenção a longo prazo ou backups em disco com replicação off-site.

\subsection{Equipamentos de Rede}
Os equipamentos de rede incluirão:
\begin{itemize}
	\item \textbf{Core switches:} configurados em redundância (Stack ou Chassis-based).
	\item \textbf{Firewalls:} operando em modo HA (High Availability).
	\item \textbf{Roteadores:} redundantes para saída à internet e interconexão com outros sites.
\end{itemize}

\chapter{Considerações de Segurança}
\section{Controle de Acesso Físico}
Controle de acesso será implementado com crachás com autenticação, biometria e CFTV (circuito fechado de TV), garantindo monitoramento 24/7.

\section{Proteção Contra Incêndio}
Sistemas de detecção precoce (Vesda) serão instalados para identificar sinais iniciais de fumaça. A supressão será feita com gás inerte (FM200, Novec 1230) ou sprinklers baseados em água com pré-ação, dependendo da área e do risco envolvido.

\section{Monitoramento e Gerenciamento}
Ferramentas de DCIM (Data Center Infrastructure Management) serão utilizadas para consolidar o monitoramento de energia, refrigeração, rede e segurança, proporcionando uma visão integrada e facilitando a gestão proativa.

\chapter{Apresentação dos Resultados}
\section{Visão Geral}
O conceito de Tier III foi escolhido devido ao equilíbrio entre custo e benefício, garantindo alta disponibilidade (99.982\%) e capacidade de manutenção simultânea. Cada redundância implementada, como N+1 em energia, redes e refrigeração, assegura continuidade operacional mesmo durante manutenções planejadas.

\section{Diagramas}
\subsection{Fluxo de Energia}
O diagrama de fluxo de energia ilustra o caminho desde a concessionária até os racks, incluindo redundâncias como múltiplas entradas de energia, UPS em configuração N+1 e geradores.

\subsection{Topologia de Rede}
A topologia de rede apresenta caminhos redundantes, com ISPs distintos, switches e roteadores configurados em alta disponibilidade (HA), além de segmentação lógica por VLANs.

\subsection{Layout Físico}
O layout físico dos racks segue o esquema de corredores quentes e frios, otimizando o fluxo de ar e a eficiência energética. Cada rack é organizado para facilitar o gerenciamento e a expansão futura.

\section{Pontos de Destaque}
\subsection{Custo-Benefício}
O Tier III foi escolhido por equilibrar alto nível de disponibilidade com investimentos mais razoáveis que um Tier IV. A redundância N+1 em sistemas críticos garante confiabilidade sem custos excessivos.

\subsection{Estratégias de Contingência}
As estratégias de contingência incluem geradores e UPS redundantes, storage com controladoras duplas e backups off-site, assegurando continuidade do negócio em situações adversas.

\section{Planos Futuros}
\subsection{Escalabilidade}
O projeto prevê a possibilidade de expansão, com mais racks e servidores, para atender demandas futuras.

\subsection{Upgrades Tecnológicos}
Planos futuros incluem migração para redes de maior velocidade (25/40/100 GbE), adoção de novas tecnologias de armazenamento e implementação de ferramentas avançadas de automação e monitoramento (DCIM/AI).
\chapter{Conclusão}
Este projeto conceitual visa um data center de alta disponibilidade com classificação Tier III, garantindo N+1 em energia, refrigeração e conectividade de rede. O layout físico com hot aisle/cold aisle, a distribuição redundante de energia e a segmentação de rede permitem manutenção sem downtime, atendendo a SLAs rigorosos. A segurança física e lógica, aliada a um sistema de backup robusto (incluindo off-site), completa o escopo de um data center confiável e escalável.

% ----------------------------------------------------------
% ELEMENTOS PÓS-TEXTUAIS
% ----------------------------------------------------------
\postextual
% ----------------------------------------------------------

% ----------------------------------------------------------
% Referências bibliográficas
% ----------------------------------------------------------
% \bibliography{referencias}

% ----------------------------------------------------------
% Glossário
% ----------------------------------------------------------
%
% Consulte o manual da classe abntex2 para orientações sobre o glossário.
%
%\glossary

%---------------------------------------------------------------------
% INDICE REMISSIVO
%---------------------------------------------------------------------
%%%%%MF\phantompart
%%%%%MF\printindex
%---------------------------------------------------------------------

\end{document}
