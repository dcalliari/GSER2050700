%% abtex2-modelo-trabalho-academico.tex, v-1.9.2 laurocesar
%% Copyright 2012-2014 by abnTeX2 group at http://abntex2.googlecode.com/ 
%%
%% This work may be distributed and/or modified under the
%% conditions of the LaTeX Project Public License, either version 1.3
%% of this license or (at your option) any later version.
%% The latest version of this license is in
%%   http://www.latex-project.org/lppl.txt
%% and version 1.3 or later is part of all distributions of LaTeX
%% version 2005/12/01 or later.
%%
%% This work has the LPPL maintenance status `maintained'.
%% 
%% The Current Maintainer of this work is the abnTeX2 team, led
%% by Lauro César Araujo. Further information are available on 
%% http://abntex2.googlecode.com/
%%
%% This work consists of the files abntex2-modelo-trabalho-academico.tex,
%% abntex2-modelo-include-comandos and abntex2-modelo-references.bib
%%

% ------------------------------------------------------------------------
% ------------------------------------------------------------------------
% abnTeX2: Modelo de Trabalho Academico (tese de doutorado, dissertacao de
% mestrado e trabalhos monograficos em geral) em conformidade com 
% ABNT NBR 14724:2011: Informacao e documentacao - Trabalhos academicos -
% Apresentacao
% ------------------------------------------------------------------------
% ------------------------------------------------------------------------

%-------------------------------------------------------------------------
% Modelo adaptado especificamente para o contexto do PPgSI-EACH-USP por 
% Marcelo Fantinato, com auxílio dos Professores Norton T. Roman, Helton
% H. Bíscaro e Sarajane M. Peres, em 2015, com muitos agradecimentos aos 
% criadores da classe e do modelo base.
%
% 20/06/2017: inclusão de "lista de quadros" com base no especificado em:
% https://github.com/abntex/abntex2/wiki/HowToCriarNovoAmbienteListing,
% de autoria de "Eduardo de Santana Medeiros Alexandre".
%
%-------------------------------------------------------------------------

\documentclass[
	% -- opções da classe memoir --
	12pt,				% tamanho da fonte
	% openright,			% capítulos começam em pág ímpar (insere página vazia caso preciso)
	oneside,			% para impressão apenas no anverso (apenas frente). Oposto a twoside
	a4paper,			% tamanho do papel. 
	% -- opções da classe abntex2 --
	%chapter=TITLE,		% títulos de capítulos convertidos em letras maiúsculas
	%section=TITLE,		% títulos de seções convertidos em letras maiúsculas
	%subsection=TITLE,	% títulos de subseções convertidos em letras maiúsculas
	%subsubsection=TITLE,% títulos de subsubseções convertidos em letras maiúsculas
	% -- opções do pacote babel --
	english,			% idioma adicional para hifenização
	%french,				% idioma adicional para hifenização
	%spanish,			% idioma adicional para hifenização
	brazil				% o último idioma é o principal do documento
	]{abntex2unama}

% ---
% Pacotes básicos 
% ---
\usepackage{lmodern}			% Usa a fonte Latin Modern			
\usepackage[T1]{fontenc}		% Selecao de codigos de fonte.
\usepackage[utf8]{inputenc}		% Codificacao do documento (conversão automática dos acentos)
\usepackage{lastpage}			% Usado pela Ficha catalográfica
\usepackage{indentfirst}		% Indenta o primeiro parágrafo de cada seção.
\usepackage{color}				% Controle das cores
\usepackage{graphicx}			% Inclusão de gráficos
\usepackage{microtype} 			% para melhorias de justificação
\usepackage{pdfpages}     %para incluir pdf
\usepackage{algorithm}			%para ilustrações do tipo algoritmo
\usepackage{mdwlist}			%para itens com espaço padrão da abnt
\usepackage[noend]{algpseudocode}			%para ilustrações do tipo algoritmo
		
% ---
% Pacotes adicionais, usados apenas no âmbito do Modelo Canônico do abnteX2
% ---
\usepackage{lipsum}				% para geração de dummy text
% ---

% ---
% Pacotes de citações
% ---
\usepackage{hyperref}
\usepackage[brazilian,hyperpageref]{backref}	 % Paginas com as citações na bibl
\usepackage[alf,abnt-etal-list=0,abnt-etal-text=it]{abntex2cite}	% Citações padrão ABNT

% --- 
% CONFIGURAÇÕES DE PACOTES
% --- 

% ---
% Configurações do pacote backref
% Usado sem a opção hyperpageref de backref
\renewcommand{\backrefpagesname}{Citado na(s) página(s):~}
% Texto padrão antes do número das páginas
\renewcommand{\backref}{}
% Define os textos da citação
\renewcommand*{\backrefalt}[4]{
	\ifcase #1 %
		Nenhuma citação no texto.%
	\or
		Citado na página #2.%
	\else
		Citado #1 vezes nas páginas #2.%
	\fi}%
% ---

% ---
% Informações de dados para CAPA e FOLHA DE ROSTO
% ---

%-------------------------------------------------------------------------
% Comentário adicional do PPgSI - Informações sobre o ``instituicao'':
%
% Não mexer. Deixar exatamente como está.
%
%-------------------------------------------------------------------------
\instituicao{
	UNIVERSIDADE DA AMAZÔNIA
	\par
	CURSO DE BACHARELADO EM CIÊNCIA DA COMPUTAÇÃO
	\par
	DISCIPLINA DE INFRAESTRUTURA DE DATACENTERS
	}

%-------------------------------------------------------------------------
% Comentário adicional do PPgSI - Informações sobre o ``título'':
%
% Em maiúscula apenas a primeira letra da sentença (do título), exceto 
% nomes próprios, geográficos, institucionais ou Programas ou Projetos ou 
% siglas, os quais podem ter letras em maiúscula também.
%
% O subtítulo do trabalho é opcional.
% Sem ponto final.
%
% Atenção: o título da Dissertação/Tese na versão corrigida não pode mudar. 
% Ele deve ser idêntico ao da versão original.
%
%-------------------------------------------------------------------------
\titulo{Avaliar o Impacto das Normas e Boas Práticas em um Data Center Fictício}

%-------------------------------------------------------------------------
% Comentário adicional do PPgSI - Informações sobre o ``autor'':
%
% Todas as letras em maiúsculas.
% Nome completo.
% Sem ponto final.
% Para adicionar mais de um autor, basta acrescentar \\ entre os autores.
%-------------------------------------------------------------------------
\autor{\uppercase{Daniel Bahia Pinheiro Calliari}}

%-------------------------------------------------------------------------
% Comentário adicional do PPgSI - Informações sobre o ``local'':
%
% Não incluir o ``estado''.
% Sem ponto final.
%-------------------------------------------------------------------------
\local{Belém}

%-------------------------------------------------------------------------
% Comentário adicional do PPgSI - Informações sobre a ``data'':
%
% Colocar o ano do depósito (ou seja, o ano da entrega) da respectiva 
% versão, seja ela a versão original (para a defesa) seja ela a versão 
% corrigida (depois da aprovação na defesa). 
%
% Atenção: Se a versão original for depositada no final do ano e a versão 
% corrigida for entregue no ano seguinte, o ano precisa ser atualizado no 
% caso da versão corrigida. 
% Cuidado, pois o ano da ``capa externa'' também precisa ser atualizado 
% nesse caso.
%
% Não incluir o dia, nem o mês.
% Sem ponto final.
%-------------------------------------------------------------------------
\data{2025}

%-------------------------------------------------------------------------
% Comentário adicional do PPgSI - Informações sobre o ``Orientador'':
%
% Se for uma professora, trocar por ``Profa. Dra.''
% Nome completo.
% Sem ponto final.
%-------------------------------------------------------------------------
\orientador{Prof. Dr. Fulano de Tal}

%-------------------------------------------------------------------------
% Comentário adicional do PPgSI - Informações sobre o ``Coorientador'':
%
% Opcional. Incluir apenas se houver co-orientador formal, de acordo com o 
% Regulamento do Programa.
%
% Se for uma professora, trocar por ``Profa. Dra.''
% Nome completo.
% Sem ponto final.
%-------------------------------------------------------------------------
\coorientador{Prof. Dr. Fulano de Tal}

\tipotrabalho{Dissertação (Mestrado) / Tese (Doutorado)}

\preambulo{
%-------------------------------------------------------------------------
% Comentário adicional do PPgSI - Informações sobre o texto ``Versão 
% original'':
%
% Não usar para Qualificação.
% Não usar para versão corrigida de Dissertação/Tese.
%
%-------------------------------------------------------------------------
Versão original \newline \newline \newline 
%-------------------------------------------------------------------------
% Comentário adicional do PPgSI - Informações sobre o ``texto principal do
% preambulo'':
%
% Para Doutorado, trocar por: Tese apresentada à Escola de Artes, Ciências e Humanidades da Universidade de São Paulo para obtenção do título de Doutor (ou Doutora) em Ciências pelo Programa de Pós-graduação em Sistemas de Informação. 
%
% Para Qualificação, trocar por: Projeto de pesquisa para exame de qualificação apresentado à Escola de Artes, Ciências e Humanidades da Universidade de São Paulo como parte dos requisitos para obtenção do título de Mestre (ou Doutor ou Doutora) em Ciências pelo Programa de Pós-graduação em Sistemas de Informação.
%
%-------------------------------------------------------------------------
Trabalho acadêmico apresentado à Universidade da Amazônia como parte de disciplinas do Curso de Bacharelado em Ciência da Computação.
%
\newline \newline Área de concentração: Infraestrutura de Datacenters
%-------------------------------------------------------------------------
% Comentário adicional do PPgSI - Informações sobre o texto da ``Versão 
% corrigida'':
%
% Não usar para Qualificação.
% Não usar para versão original de Dissertação/Tese.
% 
% Substituir ``xx de xxxxxxxxxxxxxxx de xxxx'' pela ``data da defesa''.
%
%-------------------------------------------------------------------------
% \newline \newline \newline Versão corrigida contendo as alterações solicitadas pela comissão julgadora em xx de xxxxxxxxxxxxxxx de xxxx. A versão original encontra-se em acervo reservado na Biblioteca da EACH-USP e na Biblioteca Digital de Teses e Dissertações da USP (BDTD), de acordo com a Resolução CoPGr 6018, de 13 de outubro de 2011.
}
% ---


% ---
% Configurações de aparência do PDF final

% alterando o aspecto da cor azul
\definecolor{blue}{RGB}{41,5,195}

% informações do PDF
\makeatletter
\hypersetup{
     	%pagebackref=true,
		pdftitle={\@title}, 
		pdfauthor={\@author},
    	pdfsubject={\imprimirpreambulo},
	    pdfcreator={laTeX com abnTeX2 adaptado para a UNAMA/ALC},
		pdfkeywords={abnt}{latex}{abntex}{abntex2unama}{trabalho acadêmico}{unama}, 
		colorlinks=true,       		% false: boxed links; true: colored links
    	linkcolor=blue,          	% color of internal links
    	citecolor=blue,        		% color of links to bibliography
    	filecolor=magenta,      		% color of file links
		urlcolor=blue,
		bookmarksdepth=4
}
\makeatother
% --- 

% --- 
% Espaçamentos entre linhas e parágrafos 
% --- 

% O tamanho do parágrafo é dado por:
\setlength{\parindent}{1.25cm}

% Controle do espaçamento entre um parágrafo e outro:
\setlength{\parskip}{0cm}  % tente também \onelineskip
\renewcommand{\baselinestretch}{1.5}

% ---
% compila o indice
% ---
\makeindex
% ---

	% Controlar linhas orfas e viuvas
  \clubpenalty10000
  \widowpenalty10000
  \displaywidowpenalty10000

% ----
% Início do documento
% ----
\begin{document}

% Retira espaço extra obsoleto entre as frases.
\frenchspacing

% ----------------------------------------------------------
% ELEMENTOS PRÉ-TEXTUAIS
% ----------------------------------------------------------
% \pretextual

% ---
% Capa
% ---
%-------------------------------------------------------------------------
% Comentário adicional do PPgSI - Informações sobre a ``capa'':
%
% Esta é a ``capa'' principal/oficial do trabalho, a ser impressa apenas 
% para os casos de encadernação simples (ou seja, em ``espiral'' com 
% plástico na frente).
% 
% Não imprimir esta ``capa'' quando houver ``capa dura'' ou ``capa brochura'' 
% em que estas mesmas informações já estão presentes nela.
%
%-------------------------------------------------------------------------
\imprimircapa
% ---

% ---
% inserir o sumario
% ---
\pdfbookmark[0]{\contentsname}{toc}
\tableofcontents*
\cleardoublepage
% ---



% ----------------------------------------------------------
% ELEMENTOS TEXTUAIS
% ----------------------------------------------------------
\textual



%-------------------------------------------------------------------------
% Comentário adicional do PPgSI - Informações sobre ``títulos de seções''
% 
% Para todos os títulos (seções, subseções, tabelas, ilustrações, etc.):
%
% Em maiúscula apenas a primeira letra da sentença (do título), exceto 
% nomes próprios, geográficos, institucionais ou Programas ou Projetos ou
% siglas, os quais podem ter letras em maiúscula também.
%
%-------------------------------------------------------------------------
\chapter{Objetivo}
Avaliar o impacto das normas e boas práticas em um data center fictício, identificando os requisitos operacionais para um data center Tier III, propondo soluções para eficiência energética e segurança, e apresentando um plano de manutenção preventiva. O projeto visa garantir a continuidade e eficiência das operações através da análise detalhada dos requisitos de redundância, manutenibilidade e resiliência, bem como a implementação de soluções inovadoras para gestão energética e segurança dos dados e infraestrutura.

\chapter{Requisitos Operacionais para um Data Center Tier III}

\section{Redundância}
\begin{itemize}
	\item \textbf{Componentes Redundantes}: Implementação de componentes redundantes em todos os sistemas críticos
	\item \textbf{Fontes de Energia}: Múltiplas fontes de energia independentes para garantir continuidade
	\item \textbf{Sistemas de Backup}: UPS e geradores com redundância N+1
\end{itemize}

\section{Tempo de Atividade}
\begin{itemize}
	\item \textbf{Disponibilidade}: Garantia de 99.982\% de uptime
	\item \textbf{Downtime Máximo}: Limite de 1.6 horas de downtime anual
	\item \textbf{Monitoramento Contínuo}: Sistemas de monitoramento 24/7
\end{itemize}

\section{Manutenibilidade}
\begin{itemize}
	\item \textbf{Manutenção Simultânea}: Capacidade de realizar manutenção sem interrupção das operações
	\item \textbf{Acessibilidade}: Acesso facilitado a todos os componentes críticos
	\item \textbf{Procedimentos Documentados}: Documentação detalhada dos procedimentos de manutenção
\end{itemize}

\section{Capacidade de Resiliência}
\begin{itemize}
	\item \textbf{Tolerância a Falhas}: Suporte a falhas sem impacto nas operações
	\item \textbf{Isolamento de Problemas}: Capacidade de isolar componentes com falha
	\item \textbf{Recuperação Automática}: Sistemas de failover automático
\end{itemize}

\chapter{Soluções para Eficiência Energética}

\section{Uso de Energia Renovável}
\begin{itemize}
	\item \textbf{Energia Solar}: Integração de painéis solares
	\item \textbf{Energia Eólica}: Avaliação do potencial de energia eólica
	\item \textbf{Fontes Alternativas}: Redução da dependência de fontes não-renováveis
\end{itemize}

\section{Sistemas de Resfriamento Eficientes}
\begin{itemize}
	\item \textbf{Resfriamento por Água Gelada}: Implementação de sistemas eficientes
	\item \textbf{Corredores Quentes e Frios}: Otimização do fluxo de ar
	\item \textbf{Free Cooling}: Aproveitamento das condições ambientais
\end{itemize}

\section{Virtualização de Servidores}
\begin{itemize}
	\item \textbf{Consolidação}: Redução do hardware físico
	\item \textbf{Otimização}: Melhor aproveitamento dos recursos
	\item \textbf{Escalabilidade}: Flexibilidade na gestão de recursos
\end{itemize}

\section{Monitoramento de Energia}
\begin{itemize}
	\item \textbf{Medição em Tempo Real}: Monitoramento contínuo do consumo
	\item \textbf{Análise de Dados}: Identificação de desperdícios
	\item \textbf{Otimização Contínua}: Ajustes baseados em métricas
\end{itemize}

\chapter{Soluções para Segurança}

\section{Segurança Física}
\begin{itemize}
	\item \textbf{Controle de Acesso}: Sistemas biométricos e cartões
	\item \textbf{Vigilância}: Monitoramento 24/7
	\item \textbf{Barreiras Físicas}: Proteção contra acesso não autorizado
\end{itemize}

\section{Segurança de Dados}
\begin{itemize}
	\item \textbf{Criptografia}: Proteção dos dados armazenados e em trânsito
	\item \textbf{Firewalls}: Controle de tráfego de rede
	\item \textbf{Sistemas IDS/IPS}: Detecção e prevenção de intrusões
\end{itemize}

\section{Planos de Recuperação de Desastres}
\begin{itemize}
	\item \textbf{Backups}: Procedimentos regulares de backup
	\item \textbf{Testes}: Verificações periódicas dos planos
	\item \textbf{Continuidade}: Estratégias para manter operações
\end{itemize}

\section{Treinamento de Funcionários}
\begin{itemize}
	\item \textbf{Capacitação}: Treinamentos regulares
	\item \textbf{Conscientização}: Boas práticas de segurança
	\item \textbf{Atualização}: Revisão periódica dos procedimentos
\end{itemize}

\chapter{Plano de Manutenção Preventiva}

\section{Manutenção de Hardware}
\begin{itemize}
	\item \textbf{Verificações Regulares}: Inspeção periódica de componentes
	\item \textbf{Limpeza de Equipamentos}: Remoção de poeira e detritos
	\item \textbf{Substituição de Componentes}: Troca preventiva de peças desgastadas
\end{itemize}

\section{Manutenção de Software}
\begin{itemize}
	\item \textbf{Atualizações}: Aplicação de patches e atualizações
	\item \textbf{Backups}: Rotinas de backup programadas
	\item \textbf{Monitoramento}: Sistemas de detecção de problemas
\end{itemize}

\section{Manutenção de Infraestrutura}
\begin{itemize}
	\item \textbf{Inspeções}: Verificação regular dos sistemas críticos
	\item \textbf{Testes de Redundância}: Validação dos sistemas backup
	\item \textbf{Treinamento}: Capacitação contínua da equipe
\end{itemize}

\chapter{Conclusão}
A implementação de normas e boas práticas em um data center Tier III é fundamental para garantir sua operação eficiente e segura. O projeto apresentado demonstrou a importância de requisitos como redundância N+1, disponibilidade de 99.982\%, manutenibilidade concorrente e alta resiliência. As soluções propostas para eficiência energética, incluindo energia renovável e sistemas otimizados de resfriamento, contribuem para redução de custos e sustentabilidade.

As medidas de segurança física e lógica, em conjunto com o plano de manutenção preventiva, asseguram a proteção dos ativos e a continuidade das operações. A adoção dessas práticas resulta em um data center mais confiável, eficiente e preparado para os desafios atuais de processamento e armazenamento de dados.

% ----------------------------------------------------------
% ELEMENTOS PÓS-TEXTUAIS
% ----------------------------------------------------------
\postextual
% ----------------------------------------------------------

% ----------------------------------------------------------
% Referências bibliográficas
% ----------------------------------------------------------
% \bibliography{referencias}

% ----------------------------------------------------------
% Glossário
% ----------------------------------------------------------
%
% Consulte o manual da classe abntex2 para orientações sobre o glossário.
%
%\glossary

%---------------------------------------------------------------------
% INDICE REMISSIVO
%---------------------------------------------------------------------
%%%%%MF\phantompart
%%%%%MF\printindex
%---------------------------------------------------------------------

\end{document}
